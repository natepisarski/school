\documentclass{article}
\usepackage{amsthm}
\usepackage{amstext}
\usepackage{amssymb}
\usepackage{scrextend}
\begin{document}
\section*{Assignment - Question 9}
Prove by Cases:\\
For all integers n, $n^2-3n+7$ is odd.\\
\textbf{Formal Restatement:}
$\forall n \in \mathbb{Z}, n is odd$
\begin{proof}
  \begin{addmargin}{1em}
    Consider that for any value of n, it is either even or odd. This is the only cause of variation in the evaluation of this expression. I will proceed by cases.\\
    \subsection*{Case Even}
    \begin{addmargin}{1em}
      Let x be any even integer.\\
      Substituting the definition of an even number (2k where k is an integer) into x for this expression evaluates to:\\
      $4k^2-6k+7$. Rewriting it as $2(2k^2-3k+3)+1$, an equivalent form, satisfies the conditions of an odd number.\\
    \end{addmargin}
    \subsection*{Case Odd}
    \begin{addmargin}{1em}
      Let x be any odd integer.\\
      Substituting the definition of an odd number (2k+1 where k is an integer) into x for the given expression evaluates to:\\
      $4k^2-2k+5$, an equivalent form of $12k^2-6k+14+1$.\\
      This can be written with the equivalent statement: $2(6k^2-3k+6)+1$, which satisfies the definition of an odd number.\\\\
    \end{addmargin}
    In all possible cases, the argument is true. The argument is valid.\\ \textbf{done}
  \end{addmargin}
\end{proof}
\end{document}
